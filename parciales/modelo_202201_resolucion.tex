\input{../template}
\setlength{\parindent}{0pt}

\title{Modelo Primer Parcial}
\author{TADs y especificación}
\date{Septiembre 2022}

\begin{document}

\maketitle

\section{Ejercicio 1}

tieneValor: $dicc(\alpha, nat) \times nat \rightarrow bool$

tieneValor(d, n) $\equiv $
\begin{lstlisting}
    if vacio?(claves(d))
        then false
        else if obtener(d, dameUno(claves(d))) = n
            then true
            else tieneValor(borrar(DameUno(claves(d)), d), n)
\end{lstlisting}
\section{Ejercicio 2}

diasRestantesDeAlquiler(inaugurarVC(cs,ps), c2) $\equiv$ vacio

diasRestantesDeAlquiler(alquilar(v,c,p), c2) $\equiv$ 
\begin{lstlisting}
    if c = c2
        then definir(p, obtener(p, pelis(v)), diasRestantesDeAlquiler(v,c))
        else diasRestantesDeAlquiler(v, c2)
\end{lstlisting}

diasRestantesDeAlquiler(devolver(v,c,p), c2) $\equiv$ 
\begin{lstlisting}
    if c = c2
        then borrar(p, diasRestantesDeAlquiler(v,c))
        else diasRestantesDeAlquiler(v,c)
\end{lstlisting}

diasRestantesDeAlquiler(pasarDia(v), c2) $\equiv$ restarUno(quitarVencidas(diasRestantesDeAlquiler(v, c2)))

clientes(inaugurarVC(cs, ps)) $\equiv$ cs

clientes(alquilar(v,c,p)) $\equiv$ clientes(v)

clientes(devolver(v,c,p)) $\equiv$ clientes(v)

clientes(pasarDia(v)) $\equiv$ clientesSinVencidas(c, clientes(v))

quitarVencidas(alqs) $\equiv$ borrarLasVencidas(alqs, claves(alqs))

borrarLasVencidas(d, c) $\equiv$ 
\begin{lstlisting}
    if vacio?(c)
        then vacio
        else if obtener(dameUno(c), d) = 0
            then borrarLasVencidas(borrar(dameUno(c), d), c)
            else definir(dameUno(c), obtener(dameUno(c), d), borrarLasVencidas(d, sinUno(c)))
\end{lstlisting}

clientesSinVencidas(v, cs) $\equiv$ 
\begin{lstlisting}[mathescape=true]
    if vacio?(cs)
        then $\emptyset$
        else if tieneValor(diasRestantesDeAlquiler(v, dameUno(cs)),0)
            then clientesSinVencidas(v, sinUno(cs))
            else Ag(dameUno(cs), clientesSinVencidas(v, sinUno(cs))
\end{lstlisting}

\section{Ejercicio 3}

\subsection*{Pregunta a}
Dos juegos de calamares se pueden diferenciar observando:
\begin{itemize}
    \item Los jugadores vivos
    \item Los seguidores que tiene cada jugador
\end{itemize}
    
\subsection*{Pregunta b}

TAD CALAMAR es String

\textbf{Observadores básicos}
\begin{itemize}
    \item jugadores: jc $\rightarrow$ conj(calamar)
    \item seguidores: calamar c $\times$ jc j $\rightarrow$ conj(calamar) \{c $\in$ jugadores(j)\}
\end{itemize}


\textbf{Igualdad observacional}
\begin{align*}
(\forall j, j': jc)(j =_{obs} j' \iff &jugadores(j) =_{obs} jugadores(j') \\
                    \wedge &(\forall c: calamar)(c \in jugadores(j) \implies_{L} seguidores(c,j) = seguidores(c, j'))
\end{align*}
   
\subsection*{Pregunta c}

\textbf{Generadores}
\begin{itemize}
    \item iniciarJuego: conj(calamates) $\rightarrow$ jc
    \item seguir: calamar a $\times$ calamar b $\times$ jc j $\rightarrow$ jc 
    
    \{a $\in$ jugadores(j) $\wedge$ b $\in$ jugadores(j) $\wedge$ $\neg$ esSeguidor(a, j)\}
    \item luchar: calamar a $\times$ calamar b $\times$ jc j $\rightarrow$ jc 
    
    \{a $\in$ jugadores(j) $\wedge$ b $\in$ jugadores(j)\}
\end{itemize}

\subsection*{Pregunda d}

jugadores(iniciarJuego(cs)) $\equiv$ cs

jugadores(seguir(a, b, j)) $\equiv$ jugadores(j)

jugadores(luchar(a, b, j)) $\equiv$ jugadores(j) - b

seguidores(c, iniciarJuego(cs)) $\equiv \emptyset$

seguidores(c, seguir(a,b,j)) $\equiv$

\begin{lstlisting}
    if c = b
        then Ag(a, seguidores(c,j))
        else seguidores(c,j)
\end{lstlisting}

seguidores(c, luchar(a,b,j)) $\equiv$
\begin{lstlisting}[mathescape=true]
    if c = a
        then seguidores(b, j) $\cup$ seguidores(c, j)
        else if c = b
            then $\emptyset$
            else seguidores(c,j)
\end{lstlisting}

\end{document}